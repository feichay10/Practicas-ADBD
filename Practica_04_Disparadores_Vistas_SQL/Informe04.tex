\documentclass{report}

% Paquetes y configuraciones adicionales
\usepackage{graphicx}
\usepackage[export]{adjustbox}
\usepackage{caption}
\usepackage{float}
\usepackage{titlesec}
\usepackage{geometry}
\usepackage[hidelinks]{hyperref}
\usepackage{titling}
\usepackage{titlesec}
\usepackage{parskip}
\usepackage{wasysym}
\usepackage{tikzsymbols}
\usepackage[spanish]{babel}

\newcommand{\subtitle}[1]{
  \posttitle{
    \par\end{center}
    \begin{center}\large#1\end{center}
    \vskip0.5em}
}

% Configura los márgenes
\geometry{
    left=2cm,   % Ajusta este valor al margen izquierdo deseado
    right=2cm,  % Ajusta este valor al margen derecho deseado
    top=3cm,
    bottom=3cm,
}

% Configuración de los títulos de las secciones
\titlespacing{\section}{0pt}{\parskip}{\parskip}
\titlespacing{\subsection}{0pt}{\parskip}{\parskip}
\titlespacing{\subsubsection}{0pt}{\parskip}{\parskip}

% Redefinir el formato de los capítulos y añadir un punto después del número
\makeatletter
\renewcommand{\@makechapterhead}[1]{%
  \vspace*{0\p@} % Ajusta este valor para el espaciado deseado antes del título del capítulo
  {\parindent \z@ \raggedright \normalfont
    \ifnum \c@secnumdepth >\m@ne
        \huge\bfseries \thechapter.\ % Añade un punto después del número
    \fi
    \interlinepenalty\@M
    #1\par\nobreak
    \vspace{10pt} % Ajusta este valor para el espacio deseado después del título del capítulo
  }}
\makeatother

% Configura para que cada \chapter no comience en una pagina nueva
% \makeatletter
% \renewcommand\chapter{\@startsection{chapter}{0}{\z@}%
%     {-3.5ex \@plus -1ex \@minus -.2ex}%
%     {2.3ex \@plus.2ex}%
%     {\normalfont\Large\bfseries}}
% \makeatother

\begin{document}

% Portada del informe
\title{Práctica 04. Disparadores y Vistas en SQL}
\subtitle{Administracion y Diseño de Bases de Datos}
\author{Cheuk Kelly Ng Pante}
\date{\today}

\maketitle

% Índice
\tableofcontents

% Nueva página para el primer capítulo
\cleardoublepage

% Secciones del informe
% Capitulo 1
\chapter{Realizar la restauración de la base de datos alquiler.tar}
Para realizar la restauración de la base de datos \emph{alquiler.tar}, primero debemos crear la base de datos \emph{ALQUILERDVD} y luego restaurar la base de datos con el comando \emph{pg\_restore}, como se muestra a continuación:
\begin{verbatim}
usuario@ubuntu# pg_restore -U postgres -d alquilerdvd ./alquiler.tar
\end{verbatim}

\begin{figure}[H]
  \centering
  \includegraphics[scale=0.55]{img/restore_DB.png}
  \caption{Restauración de la base de datos}
  \label{fig:restauración de la base de datos}
\end{figure}

% Capitulo 2
\chapter{Identificacion de las tablas, vistas y secuencias}
Para identificar las tablas, vistas y secuencias de la base de datos \emph{ALQUILERDVD}, hay que usar la terminal interactiva de \emph{PostgreSQL} y ejecutar los siguiente comandos:
\begin{verbatim}
usuario@ubuntu# sudo -u postgres psql
postgres=# \c alquilerdvd 
alquilerdvd=# \dt
alquilerdvd=# \dv
alquilerdvd=# \ds
\end{verbatim}

\begin{figure}[H]
  \centering
  \includegraphics[scale=0.48]{img/tablas_vistas_secuencias.png}
  \caption{Identificación de las tablas, vistas y secuencias}
  \label{fig:identificación de las tablas, vistas y secuencias}
\end{figure}

% Capitulo 3
\chapter{Identifique las tablas principales y sus principales elementos}

% Tabla: actor
\CIRCLE \ \ \textbf{Tabla:} \emph{actor}
\begin{itemize}
  \item \textbf{Descripción:} Contiene la información de los actores.
  \item \textbf{Elementos:} \emph{actor\_id, first\_name, last\_name, last\_update}
\end{itemize}

% Tabla: address
\CIRCLE \ \ \textbf{Tabla:} \emph{address}
\begin{itemize}
  \item \textbf{Descripción:} Contiene la información de las direcciones.
  \item \textbf{Elementos:} \emph{address\_id, address, address2, district, city\_id, postal\_code, phone, last\_update}
\end{itemize}

% Tabla: category
\CIRCLE \ \ \textbf{Tabla:} \emph{category}
\begin{itemize}
  \item \textbf{Descripción:} Contiene la información de las categorías.
  \item \textbf{Elementos:} \emph{category\_id, name, last\_update}
\end{itemize}

% Tabla: city
\CIRCLE \ \ \textbf{Tabla:} \emph{city}
\begin{itemize}
  \item \textbf{Descripción:} Contiene la información de las ciudades.
  \item \textbf{Elementos:} \emph{city\_id, city, country\_id, last\_update}
\end{itemize}

% Tabla: country
\CIRCLE \ \ \textbf{Tabla:} \emph{country}
\begin{itemize}
  \item \textbf{Descripción:} Contiene la información de los países.
  \item \textbf{Elementos:} \emph{country\_id, country, last\_update}
\end{itemize}

% Tabla: customer
\CIRCLE \ \ \textbf{Tabla:} \emph{customer}
\begin{itemize}
  \item \textbf{Descripción:} Contiene la información de los clientes.
  \item \textbf{Elementos:} \emph{customer\_id, store\_id, first\_name, last\_name, email, address\_id, activebool, create\_date, last\_update, active}
\end{itemize}

% Tabla: film
\CIRCLE \ \ \textbf{Tabla:} \emph{film}
\begin{itemize}
  \item \textbf{Descripción:} Contiene la información de las películas.
  \item \textbf{Elementos:} \emph{film\_id, title, description, release\_year, language\_id, rental\_duration, rental\_rate, length, replacement\_cost, rating, last\_update, special\_features, fulltext}
\end{itemize}

% Tabla: film_actor
\CIRCLE \ \ \textbf{Tabla:} \emph{film\_actor}
\begin{itemize}
  \item \textbf{Descripción:} Contiene la información de los actores de las películas.
  \item \textbf{Elementos:} \emph{actor\_id, film\_id, last\_update}
\end{itemize}

% Tabla: film_category
\CIRCLE \ \ \textbf{Tabla:} \emph{film\_category}
\begin{itemize}
  \item \textbf{Descripción:} Contiene la información de las categorías de las películas.
  \item \textbf{Elementos:} \emph{film\_id, category\_id, last\_update}
\end{itemize}

% Nueva página 
\cleardoublepage

% Tabla: inventory
\CIRCLE \ \ \textbf{Tabla:} \emph{inventory}
\begin{itemize}
  \item \textbf{Descripción:} Contiene la información de los inventarios.
  \item \textbf{Elementos:} \emph{inventory\_id, film\_id, store\_id, last\_update}
\end{itemize}

% % Nueva página 
% \cleardoublepage

% Tabla: language
\CIRCLE \ \ \textbf{Tabla:} \emph{language}
\begin{itemize}
  \item \textbf{Descripción:} Contiene la información de los lenguajes.
  \item \textbf{Elementos:} \emph{language\_id, name, last\_update}
\end{itemize}

% Tabla: payment
\CIRCLE \ \ \textbf{Tabla:} \emph{payment}
\begin{itemize}
  \item \textbf{Descripción:} Contiene la información de los pagos.
  \item \textbf{Elementos:} \emph{payment\_id, customer\_id, staff\_id, rental\_id, amount, payment\_date}
\end{itemize}

% Tabla: rental
\CIRCLE \ \ \textbf{Tabla:} \emph{rental}
\begin{itemize}
  \item \textbf{Descripción:} Contiene la información de los alquileres.
  \item \textbf{Elementos:} \emph{rental\_id, rental\_date, inventory\_id, customer\_id, return\_date, staff\_id, last\_update}
\end{itemize}

% Tabla: staff
\CIRCLE \ \ \textbf{Tabla:} \emph{staff}
\begin{itemize}
  \item \textbf{Descripción:} Contiene la información de los empleados.
  \item \textbf{Elementos:} \emph{staff\_id, first\_name, last\_name, address\_id, email, store\_id, active, username, password, last\_update, picture}
\end{itemize}

% Tabla: store
\CIRCLE \ \ \textbf{Tabla:} \emph{store}
\begin{itemize}
  \item \textbf{Descripción:} Contiene la información de las tiendas.
  \item \textbf{Elementos:} \emph{store\_id, manager\_staff\_id, address\_id, last\_update}
\end{itemize}

% Capitulo 4
\chapter{Realizacion de consultas}

% Consulta a)
\section{Ventas totales por categoria}
Para obtener las ventas totales por categoría de películas ordenadas descendentemente, se debe ejecutar la siguiente consulta:
\begin{verbatim}
alquilerdvd=# SELECT category.name, SUM(payment.amount) AS total_sales
              FROM category
              INNER JOIN film_category ON category.category_id = film_category.category_id
              INNER JOIN film ON film_category.film_id = film.film_id
              INNER JOIN inventory ON film.film_id = inventory.film_id
              INNER JOIN rental ON inventory.inventory_id = rental.inventory_id
              INNER JOIN payment ON rental.rental_id = payment.rental_id
              GROUP BY category.name
              ORDER BY total_sales DESC;
\end{verbatim}
\begin{figure}[H]
  \centering
  \includegraphics[scale=0.48]{img/querie_a.png}
  \caption{Ventas totales por categoria}
  \label{fig:ventas totales por categoria}
\end{figure}

% Nueva página 
\cleardoublepage

% Consulta b)
\section{Ventas totales por tienda}
Para obtener las ventas totales por tienda, donde se refleje la ciudad, el país
(concatenar la ciudad y el país empleando como separador la “,”), y el
encargado. Pusiera emplear GROUP BY, ORDER BY
\begin{verbatim}
alquilerdvd=# SELECT CONCAT(city.city, ', ', country.country) AS city_country,
              CONCAT(staff.first_name, ' ', staff.last_name) AS manager,
              SUM(payment.amount) AS total_sales
              FROM store
              INNER JOIN staff ON store.manager_staff_id = staff.staff_id
              INNER JOIN address ON store.address_id = address.address_id
              INNER JOIN city ON address.city_id = city.city_id
              INNER JOIN country ON city.country_id = country.country_id
              INNER JOIN inventory ON store.store_id = inventory.store_id
              INNER JOIN rental ON inventory.inventory_id = rental.inventory_id
              INNER JOIN payment ON rental.rental_id = payment.rental_id
              GROUP BY city_country, manager
              ORDER BY total_sales DESC;
\end{verbatim}
\begin{figure}[H]
  \centering
  \includegraphics[scale=0.60]{img/querie_b.png}
  \caption{Ventas totales por tienda}
  \label{fig:ventas totales por tienda}
\end{figure}

% Nueva página 
\cleardoublepage

% Consulta c)
\section{Lista de películas}
Para obtener una lista de películas, donde se reflejen el identificador, el
título, descripción, categoría, el precio, la duración de la película,
clasificación, nombre y apellidos de los actores (puede realizar una
concatenación de ambos). Pusiera emplear GROUP BY
\begin{verbatim}
alquilerdvd=# SELECT film.film_id, title, description, category.name AS category_name, 
              rental_rate, length, rating,
              actor.first_name || '  ' || actor.last_name AS actor_name
              FROM film
              INNER JOIN film_actor ON film.film_id = film_actor.film_id
              INNER JOIN actor ON film_actor.actor_id = actor.actor_id
              INNER JOIN film_category ON film.film_id = film_category.film_id
              INNER JOIN category ON film_category.category_id = category.category_id
              ORDER BY film_id;
\end{verbatim}
\begin{figure}[H]
  \centering
  \includegraphics[scale=0.28]{img/querie_c.png}
  \caption{Lista de películas}
  \label{fig:lista de películas}
\end{figure}

% Nueva página 
\cleardoublepage

% Consulta d)
\section{Información de los actores}
Para obtener la información de los actores, donde se incluya sus nombres y
apellidos, las categorías y sus películas. Los actores deben de estar
agrupados y, las categorías y las películas deben estar concatenados por
“:”
\begin{verbatim}
alquilerdvd=# SELECT actor.first_name || ' ' || actor.last_name AS actor_name,
              string_agg(DISTINCT category.name, ':') AS categories,
              string_agg(DISTINCT film.title, ':') AS films
              FROM actor
              INNER JOIN film_actor ON actor.actor_id = film_actor.actor_id
              INNER JOIN film_category ON film_actor.film_id = film_category.film_id
              INNER JOIN category ON film_category.category_id = category.category_id
              INNER JOIN film ON film_category.film_id = film.film_id
              GROUP BY actor_name;
\end{verbatim}
\begin{figure}[H]
  \centering
  \includegraphics[scale=0.60]{img/querie_d.png}
  \caption{Información de los actores}
  \label{fig:información de los actores}
\end{figure}

% Nueva página
\cleardoublepage

% Capitulo 5
\chapter{Vistas de las consultas realizadas}
Para crear las vistas de las consultas realizadas se usará el prefijo \emph{view\_} para identificarlas. A continuación se muestra el código de cada vista:

% Vista a)
\section{Ventas totales por categoria}



\end{document}