\documentclass[11pt]{report}

% Paquetes y configuraciones adicionales
\usepackage{graphicx}
\usepackage[export]{adjustbox}
\usepackage{caption}
\usepackage{float}
\usepackage{titlesec}
\usepackage{geometry}
\usepackage[hidelinks]{hyperref}
\usepackage{titling}
\usepackage{titlesec}
\usepackage{parskip}
\usepackage{wasysym}
\usepackage{tikzsymbols}
\usepackage[spanish]{babel}

\newcommand{\subtitle}[1]{
  \posttitle{
    \par\end{center}
    \begin{center}\large#1\end{center}
    \vskip0.5em}
}

% Configura los márgenes
\geometry{
    left=2cm,   % Ajusta este valor al margen izquierdo deseado
    right=2cm,  % Ajusta este valor al margen derecho deseado
    top=3cm,
    bottom=3cm,
}

% Configuración de los títulos de las secciones
\titlespacing{\section}{0pt}{\parskip}{\parskip}
\titlespacing{\subsection}{0pt}{\parskip}{\parskip}
\titlespacing{\subsubsection}{0pt}{\parskip}{\parskip}

% Redefinir el formato de los capítulos y añadir un punto después del número
\makeatletter
\renewcommand{\@makechapterhead}[1]{%
  \vspace*{0\p@} % Ajusta este valor para el espaciado deseado antes del título del capítulo
  {\parindent \z@ \raggedright \normalfont
    \ifnum \c@secnumdepth >\m@ne
        \huge\bfseries \thechapter.\ % Añade un punto después del número
    \fi
    \interlinepenalty\@M
    #1\par\nobreak
    \vspace{10pt} % Ajusta este valor para el espacio deseado después del título del capítulo
  }}
\makeatother

% Configura para que cada \chapter no comience en una pagina nueva
\makeatletter
\renewcommand\chapter{\@startsection{chapter}{0}{\z@}%
    {-3.5ex \@plus -1ex \@minus -.2ex}%
    {2.3ex \@plus.2ex}%
    {\normalfont\Large\bfseries}}
\makeatother

\begin{document}

% Portada del informe
\title{Seguridad e inyección en SQL}
\subtitle{Administración y Diseño de Bases de Datos}
\author{Cheuk Kelly Ng Pante, Javier González de la Barreda Arimany y Samuel Toledo Hernández}
\date{\today}

\maketitle

% Índice
\tableofcontents

% Nueva página para el primer capítulo
\cleardoublepage

% % Sangría
% \setlength{\parindent}{1em}Texto

% % Quitar sangría
% \noindent

% % Punto
% \CIRCLE \ \ \textbf{Texto} \emph{algo}
% \begin{itemize}
%   \item \textbf{Negrita:} Texto
%   \item \textbf{Negrita:} Texto
% \end{itemize}

% \begin{verbatim}
% ... Codigo
% \end{verbatim}

% Poner una imagen
% \begin{figure}[H]
%   \centering
%   \includegraphics[scale=0.55]{img/}
%   \caption{Exportación de la base de datos en formato sql}
%   \label{fig:exportación de la base de datos en formato sql}
% \end{figure}

% Secciones del informe
% Capitulo 1
\chapter{Introducción a la seguridad de bases de datos}
\section{Introducción a las bases de datos}
Una base de datos consiste en una colección de datos interrelacionados que representan
información sobre una organización o área en particular. Estas bases de datos están
organizadas según modelos de datos que definen la estructura, almacenamiento y
manipulación de la información. El modelo principal es un modelo relacional que representa
datos a través de tablas y crea relaciones entre ellos.

Los sistemas de gestión de bases de datos (DBMS) son programas informáticos que
gestionan, modifican, consultan y crean bases de datos. Entre sus ventajas importantes se
incluyen la independencia de los datos con respecto a la aplicación respectiva, la garantía
de integridad mediante reglas de validación; seguridad de la información mediante control
de acceso; y optimización del rendimiento a través de índices/vistas.

En el diseño de bases de datos intervienen una variedad de etapas, incluido el análisis de
requisitos, el diseño conceptual, el diseño lógico y el diseño físico. Los modelos de entidad,
atributos y relaciones se utilizan en el modelo de relación de entidad para representar datos
en el diseño conceptual. El modelo relacional es la base del diseño lógico, ya que presenta
datos a través de tablas, claves y restricciones.

El lenguaje utilizado para consultar la base de datos es un conjunto de instrucciones que
permiten a los usuarios seleccionar, insertar, actualizar y eliminar datos. La mayoría de los
DBMS pueden ser compatibles con SQL, qué es el lenguaje de consulta más utilizado.

En el ámbito de la programación de bases de datos, se emplean subrutinas, que son
bloques de código encargados de llevar a cabo tareas específicas relacionadas con los
datos. Estas subrutinas pueden manifestarse como procedimientos almacenados o
funciones, diferenciándose en que los procedimientos almacenados pueden retornar varios
valores o ninguno, mientras que las funciones solo pueden devolver un valor. La ejecución
de estas subrutinas dentro del Sistema de Gestión de Bases de Datos (SGBD) contribuye a
mejorar tanto la eficiencia como la seguridad de las bases de datos

\section{Seguridad de bases de datos}
La seguridad de la base de datos abarca un conjunto de herramientas, medidas y controles
diseñados para proteger la integridad, confidencialidad y disponibilidad de los datos
almacenados en una base de datos. Estas “copias” están implementadas para evitar el
acceso no autorizado, modificaciones inapropiadas o pérdidas accidentales. En este
contexto, se cubrirán aspectos técnicos y organizativos, incluidas actividades como
autenticación, auditoría, cumplimiento normativo, gestión de riesgos, educación, etc.

La importancia de la seguridad de las bases de datos reside en su función esencial para
asegurar el funcionamiento adecuado de las organizaciones y la salvaguardia de la
privacidad de las personas. Para preservar la integridad del sistema, resulta crucial hacer
frente a amenazas como ataques internos, errores humanos, vulnerabilidades de software,
ataques de inyección SQL, ataques de denegación de servicio y la presencia de malware.

Ante las amenazas mencionadas, se sugiere adoptar las mejores prácticas de
ciberseguridad, que incluyen medidas como salvaguardar la integridad física, aplicar
controles administrativos y de acceso a la red, asegurar dispositivos y cuentas de usuarios
finales, proteger el software de bases de datos, garantizar la seguridad de servidores de
aplicaciones y web, así como implementar precauciones en las copias de respaldo y llevar a
cabo auditorías periódicas.

La seguridad de las bases de datos emerge como un tema de gran importancia y
complejidad. En este contexto, se busca resguardar la información almacenada en dichas
bases contra accesos no autorizados, alteraciones indebidas y pérdidas accidentales o
maliciosas. La protección de la base de datos involucra diversos elementos críticos como:

\begin{itemize}
\item \textbf{Datos de la base de datos:} Esta información, que abarca desde nombres y
contraseñas hasta números de tarjetas de crédito y debe mantenerse confidencial,
íntegra y accesible únicamente para usuarios autorizados.
\item \textbf{Sistema de gestión de bases de datos (DBMS):} Desempeña un papel fundamental
al crear, administrar y manipular las bases de datos. Es esencial que el DBMS esté
actualizado, configurado y protegido de manera adecuada para prevenir
vulnerabilidades y ataques.
\item \textbf{Aplicaciones asociadas:} Los programas que interactúan con la base de datos para
realizar diversas operaciones deben validar y filtrar las entradas de los usuarios.
Además, es crucial que utilicen consultas parametrizadas o procedimientos
almacenados, y limiten los permisos y privilegios de los usuarios y las bases de
datos.
\item \textbf{Servidor de base de datos físico y/o virtual y hardware subyacente:} Estos
dispositivos, ya sean físicos o virtuales, donde residen la base de datos y el DBMS,
deben contar con protección física, sistemas de copia de seguridad y recuperación,
así como medidas de seguridad de red, como firewalls, antivirus y cifrado.
\item \textbf{Infraestructura informática y/o de red para acceder a la base de datos:} Los medios a
través de los cuales usuarios y aplicaciones se comunican con la base de datos,
como ordenadores, dispositivos móviles e internet, deben garantizar la seguridad y
privacidad de las comunicaciones. Esto implica el uso de protocolos seguros,
contraseñas robustas, certificados de seguridad, entre otras medidas.
\end{itemize}

La importancia de asegurar las bases de datos radica en la necesidad de prevenir o reducir
al mínimo los riesgos y las consecuencias asociadas a los ciberataques. Estos eventos
pueden resultar en daños irreparables tanto para los datos almacenados como para las
organizaciones y los usuarios involucrados. Algunos de los ciberataques más comunes y
peligrosos pueden ser:

\begin{itemize}
\item \textbf{La inyección SQL:} Implica la introducción de código SQL malicioso en las entradas
de los usuarios para alterar o acceder a la información de la base de datos. Este tipo
de ataque puede ocasionar la pérdida o el robo de datos sensibles, la modificación o
eliminación de información crucial, la ejecución de comandos arbitrarios en el
servidor, o la revelación de datos internos o confidenciales de una organización.
\item \textbf{El robo de credenciales:} Se refiere a la obtención de contraseñas o nombres de
usuario de usuarios o aplicaciones que acceden a la base de datos. El propósito de
este ataque es suplantar identidades o llevar a cabo acciones no autorizadas, lo que
puede resultar en acceso no autorizado a los datos, alteración o borrado de
información, o la propagación de malware o virus.
\item \textbf{El ransomware:} Implica cifrar los datos de la base de datos o bloquear su acceso,
con el objetivo de exigir un rescate a cambio de su liberación. Este tipo de ataque
puede conducir a la inaccesibilidad de los datos, interrupciones en el funcionamiento
del negocio o el pago de grandes sumas de dinero
\end{itemize}

La seguridad de las bases de datos exige la implementación de un conjunto integral de
medidas técnicas, organizativas y legales destinadas a resguardar los datos de posibles
amenazas, tanto internas como externas, como hackers, empleados deshonestos, errores
humanos, fallas de hardware o software, desastres naturales y violaciones normativas.
Diversas prácticas, políticas y tecnologías pueden ser adoptadas para fortalecer la
seguridad de las bases de datos, entre las que se encuentran:
\begin{itemize}
  \item Diseñar una arquitectura de base de datos segura que separe los datos sensibles de
  los no sensibles, que minimice los puntos de acceso y que aplique el principio de
  mínimo privilegio.
  \item Implementar un sistema de gestión de bases de datos (DBMS) actualizado y
  configurado adecuadamente, que ofrezca funciones integradas de seguridad, tales
  como cifrado, control de acceso, registro de eventos y detección de anomalías.
  \item Desarrollar una aplicación segura que utilice métodos de conexión seguros, que
  evite la inyección de código malicioso, valide los datos de entrada y cifre la
  información en tránsito y en reposo.
  \item Llevar a cabo pruebas de seguridad periódicas que evalúen la vulnerabilidad de la
  base de datos e identifiquen y corrijan posibles debilidades o brechas.
  \item Impartir formación y concienciación a los usuarios y administradores de la base de
  datos acerca de las mejores prácticas de seguridad, como la utilización de
  contraseñas sólidas, el cambio regular de credenciales, el bloqueo de sesiones
  inactivas y la notificación de cualquier incidente sospechoso
\end{itemize}

% Nueva página para el primer capítulo
\cleardoublepage

\chapter{Riesgos y consecuencias de la inyección SQL}
\section{Introducción a SQL}

\end{document}